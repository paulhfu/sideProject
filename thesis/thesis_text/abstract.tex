\setstretch{1.0}
\phantomsection~\label{sec:ref}
\section*{Abstrakt}
Ein Convolutional Neurales Netzwerk zu trainieren hat in der Regel den Nachteil, dass das Rückführungssignal für Vorhersagen sich aus einer differenzierbaren Loss Funktion ergeben muss. Das ist ein Grund dafür, dass die besten Resultate für viele Probleme die mit neuronalen Netzwerken gelöst werden von methoden des überwachten Lernens kommen. Da in diesem Fall eine aussagekräftige und differenzierbare Loss Funktion auf Basis von Ground Truth instanzen definiert werden kann. Bei Unüberwachtem lernen auf der anderen Seite fällt es in der Regel schwer eine differenzierbare Loss Funktion nur auf Basis vorher definierter Regeln zu definieren.\\
Bestärkendes Lernen bietet die Möglichkeit eine Vorhersage zu evaluieren und von dieser Evaluation zu lernen wobei diese Evaluation nicht in differenzierbar sein muss. Die methode die in dieser Arbei vorgestellt wird formuliert das problem der Segmentierung von Bildern als Problem welches mit bestärkendem Lernen gelöst werden kann.\\
Mehrere algorithmen des bestärkenden Lernens wie etwa ACER oder SAC wurden an dem Problem getestet. Die zentralen neuronalen Netzwerke basieren auf graph Faltungen und treffen vorhersagen für Kanten gewichte auf kanten in Graphen welche sich aus Superpixel segmentierungen der Rohdaten ergeben. Auf diesen Vorhersagen gründend wird anschließlich mithilfe des Multicut Algorithmus eine Finale Segmentierung erstellt.\\
Die Methode worde auf einem Spiel- Datensatz getestet wobei Überwachtes Lernen sowie Unüberwachtes Lernen zum Einsatz kam.
\section*{Abstract}
Training a convolutional neural network comes with the drawback that the feedback for predictions has to come from a differentiable loss function. This is one reason why the best results stem from supervised learning where a expressive and differentiable loss can be defined using ground truth labels. Unsupervised learning does not yet provide methods to evaluate a prediction in a differentiable way such that the result is an expressive loss that can compete with a supervised loss.\\
Reinforcement learning on the other hand allows for a evaluation used for a feedback signal that does not have to be dependent on the neural networks parameters and therefore does not have to be differentiable.\\
The method proposed in this work formulates the image segmentation problem as a reinforcement learning problem and provides a basic pipeline to solve this problem.\\
Several reinforcement learning algorithms like ACER and SAC have been tested on the problem. The core networks using graph convolutions on a region adjacency graph of a superpixel segmentation where the feature representations of the nodes are based on predictions of an upstream network predicting pixel embeddings. The final predictions are edge weights of the edges in the region adjacency graph of the superpixel segmentation. Based on this edge weights the multicut algorithm provides a final segmentation.\\
The method was tested on a toy dataset where supervised and semi supervised and unsupervised approaches have been used.