\setstretch{1.0}
\phantomsection~\label{sec:ref}
\section*{Abstrakt}
Ein Convolutional Neural Network (CNN) zu trainieren hat in der Regel den Nachteil, dass das Optimierungssignal der Parameter sich aus einer differenzierbaren Loss Funktion ergeben muss. Das ist ein Grund dafür, dass die besten Resultate für viele Probleme die mit neuronalen Netzwerken gelöst werden von Methoden des überwachten Lernens kommen, da in diesem Fall eine aussagekräftige und differenzierbare Loss Funktion auf Basis von Ground Truth instanzen definiert werden kann. Gegensätzlich dazu bei unüberwachtem Lernen, wo es in der Regel schwer fällt, eine differenzierbare Loss Funktion nur auf Basis vorher definierter Regeln zu definieren die das Wissen, welches zur Herstellung von Ground Truth instanzen benötigt werden würde, formalisieren.\\
Bestärkendes Lernen bietet die Möglichkeit eine Vorhersage zu evaluieren und von dieser Evaluation zu lernen wobei keine Restriktionen an den Evaluierungsprozess existieren. Die Methode die in dieser Arbei vorgestellt wird, formuliert das Problem der Segmentierung von Bildern als Problem welches mit bestärkendem Lernen gelöst werden kann.\\
Mehrere Algorithmen des bestärkenden Lernens wie etwa ACER \cite{wang2016sample} oder SAC \cite{haarnoja2018soft} werden an dem Problem getestet. Die zentralen neuronalen Netzwerke basieren auf graph Faltungen und treffen Vorhersagen für Kantengewichte auf Kanten in Graphen, welche sich aus Superpixel Segmentierungen der Rohdaten ergeben. Auf diesen Vorhersagen gründend wird anschließlich mithilfe des Multicut Algorithmus eine finale Segmentierung erstellt.\\
Die Methode wird auf einem künstlich erstellten Datensatz getestet, wobei überwachtes Lernen sowie unüberwachtes Lernen zum Einsatz kommt.
\section*{Abstract}
Training a CNN has the drawback that the feedback signal for the parameter optimization has to come from a differentiable loss function. This is one reason why the best results stem from supervised learning where an expressive and differentiable loss can be defined using ground truth labels. Unsupervised learning does not yet provide methods to evaluate a prediction in a differentiable way such that the result is an expressive loss that can compete with a supervised loss.\\
Reinforcement learning on the other hand allows for a evaluation used for a feedback signal that does not have to be dependent on the neural networks parameters and therefore does not have to be differentiable.\\
The method proposed in this work formulates the image segmentation problem as a reinforcement learning problem and provides a basic pipeline to solve this task.\\
Several reinforcement learning algorithms like ACER \cite{wang2016sample} and SAC \cite{haarnoja2018soft} have been tested on the problem. The core networks are based on graph convolutions on a region adjacency graph (rag) of a superpixel segmentation. In this graph the representations of the nodes are based on predictions of an upstream CNN predicting pixel embeddings. The final predictions are edge weights for the edges in the rag. Based on this edge weights the multicut algorithm provides a final segmentation.\\
The method was tested on a toy dataset where supervised, semi supervised and unsupervised approaches have been used.