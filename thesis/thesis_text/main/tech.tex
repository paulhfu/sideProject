\section{Technical details}\label{sec:tech}
This section reviews some technical details of the implementation of the pipeline. Except for algorithm \ref{algo:sgs} and some more helper functions for operations on graphs that have been implemented in c++ all has been written in python under heavy use of the libraries pytorch, numpy, scipy and skimage. The c++ implementations are called from the python interpreter using the pybind11 interface together with the xtensor libraries. The GCNNs have been implemented using the library pytorch-geometric, introduced in \cite{Fey/Lenssen/2019}.\\
pytorch multiprocessing is used for parallelization and synchronization in the fashion of the A3C (section \ref{ssec::a3c}). After each update step through graph convolutions in section \ref{sec:sag_gcn} node and edge features are normalized by a Batch Normalization layer \cite{ioffe2015batch}.\\
The embedding space is $\mathbb{R}^{16}$ and the node features in section \ref{sec:sag_gcn} are in the embedding space $x_{i}^0 \in \mathbb{R}^{16}$. The number of convolution iterations in section \ref{sec:sag_gcn} is $K=5$ as well as for $M=5$. The number of convolution iterations for the subgraph GCNNs is $N=10$ because here it is important that the information in the graph is spread over all nodes because of the global pooling operation afterwards. The multilayer perceptrons in eq. (3.3-3.14) in section \ref{sec:sag_gcn} all have $1$ hidden layer. The first mlp in each of the actors convolution blocks has $in=16\cdot 2$ input features and $in\cdot 10$ output features. All the following except the last one have $in\cdot 10$ input and $in\cdot 10$ output features. The last one squashes the feature vectors again to $16$ dimensions for node and edge features. For the critics first networks that operate on the whole graph the same structure holds except, here the input features are $in=16\cdot 2 + 1$ since the actions $a_{ij}$ are concatenated to the node features $x_{i}^0$.


\subsection{Batch processing}\label{ssec:batchp}
Batching a set of irregular graphs can be achieved by converting the set of graphs into one large graph whose connected components form each graph in the batched set. Doing graph convolution on the large graph yields the same result as doing the convolution on each of the batched graphs separately with the difference that the operation in performed in parallel. The same property is used for the convolution on the subgraphs.