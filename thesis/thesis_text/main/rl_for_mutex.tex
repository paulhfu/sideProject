\section{Using RL for pixel affinity predictions}~\label{sec:rl_for_seg}

This work started with the idea of an RL setting where an agent manipulates affinities between pixel pairs. Those affinities form an input to the Mutex Wateshed algorithm (see section \ref{sec:mtx_wtsd}).\\
The edge set that the Mutex Watershed typically works with are short range attractive edges and long lange repulsive edges where the length of the repulsive edges is dependent on the object sizes in the image. That means that there should be exactly one attractive edge for every directly neighboring pair of pixels and some long range repulsive edges. Defining the problem by manipulating affinities leads to a hard task since there are simply so many.\\
RL methods learns from rewards resulting from actions and an initial state. To achieve convergence, initially there need to be some rewards of a high value, that means that the actions taken lead to a fairly good segmentation. However when starting learning a network from noise it outputs completely arbitrary actions that unlikely lead to a high reward because the action space is simply too large.\\
The RL-typical bootstrapping works only if there is a meaningful gradient in the reward signal, even for random actions. It is not uncommon in RL to have a large action space while the reward is a single scalar value. If it would be possible to calculate more meaningful rewards, say per subregion in the image, one could compute a RL loss term for each of those subregions only considering the actions that manipulated affinities within this region.\\

To downsize a image segmentation problem, it is common to work with superpixels \cite{10.1007/978-3-642-23094-3_3} rather than  with pixels. A superpixel segmentation or oversegmentation is usually achieved by watersheding or smoothing algorithms.
Using Mutex Watershed one can simply globally decrease the edge weights of the attractive edges in order to arrive at an oversegmentation.
Starting from such an oversegmentation and assuming that the ground-truth segmentation is a partitioning of the superpixels, focusing solely on merging and unsmerging superpixels would be sufficient. Concerning the Mutex Watershed algorithm, a merge of two superpixels would be done by turning all the repulsive edges between them into attractive ones and vice versa for unmerging. Therefore one needs a decision/action for every neighboring pair of superpixels.\\
However there are two problems with this, first one can only perform hard merges and unmerges which leads likely to contradictions for example consider 3 adjacent superpixels and there are 2 merges and 1 unmerge predicted. This is of course nothing Mutex Watershed cannot handle but the result is likely to be not a partitioning of the superpixels and therefore not the intended result.\\
The second issue is that for CNNs it is difficult to make predictions on affinities between adjacent superpixels due to the irregularity of the region adjacency graph of the superpixels.