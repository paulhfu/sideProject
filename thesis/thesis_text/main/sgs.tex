\section{Finding subgraphs}~\label{seg:sag_gcn}
The selection of subgraphs has the restrictions that all selected subgraphs should consist of $l$ edges and that the union of all subgraphs should cover the region adjacency graph of the superpixel segmentation.\\
Additional to that, subgraphs should overlap to a certain margin and the density in each subgraph should be high. The density of a graph $G=(V,E)$ is defined by $\frac{|E|}{|V|}$. The overlaps should not be too large such that the result is still a feasible number of subgraphs. \\
Finding the densest subgraph of size $l$ in an undirected graph $G=(V,E)$ is in general a NP-hard problem \cite{densestSg}. The implemented algorithm is a fast heuristic that leverages the properties of the region adjacency in $G$ where one can assume a relatively even density over the whole graph.\\
The heuristic starts by sampling a random edge $(ij)\in E$ that is not contained in any subgraph so far. It adds it to the new (and initially empty) subgraph $SG=(SV,SE)$ and pushes its incidental nodes to a priority queue $pq$ with starting priority value $0$ (a smaller value corresponds to a higher priority). Nodes are drawn from $pq$ until the respective subgraph has the right amount of edges.\\
Drawing a node $n$ from $pq$ is followed by iteratively verifying if there is a node $m$ s.t. $(nm)\in E$ and $m\in SV$, if yes than the edge $(nm)$ is added to $SG$. If not all to $n$ adjacent nodes where accepted and the corresponding edges added to $SG$, the priority of $n$ is decreased by the amount of edges that where added and incremented by $1$. Then $n$ is pushed back into $pq$.\\
The next iteration starts by drawing the next node from $pq$. If all elements in $pq$ where drawn without an edge being added to $SG$ and if $SG$ is still incomplete, the last drawn nodes $n$ last examined neighbor $m \not\in SV$ is added to $pq$ and  the edge $(nm)$ is added to $SG$.\\
This is repeated until $SG$ is complete and the next subgraph is started by, again drawing a random edge that is not contained in any subgraph yet.\\
This continues until the union of all subgraphs covers $G$. The worst case of this method would be tree-like subgraphs overlapping completely except for one edge. However for region adjacency graphs this is unlikely too happen and can be neglected.\\
The pseudo code for the described heuristic is given in algorithm \ref{algo:sgs}.\\
\vspace{8mm}\\
\begin{algorithm}[H]
	\KwData{$G=(V, E)$, $l$}
	\KwResult{subgraphs by sets of $l$ edges}
	Initialization:$SG = \emptyset$\;
	\While{$E\backslash SG \neq \emptyset$}{
		pq = PriorityQueue\;
		prio = 0\;
		n\_draws = 0\;
		$sg = \emptyset$\;
		$sg_{vtx} = \emptyset$\;
		$i, j = (ij)$ s.t. $(ij)\in E\backslash SG$\;
		pq.push($i$, prio)\;
		pq.push($j$, prio)\;
		$sg = sg \cup (ij)$\;
		$sg_{vtx} = sg_{vtx} \cup i$\;
		$sg_{vtx} = sg_{vtx} \cup j$\;
		\While{|sg| < $l$}{
			$n$, n\_prio = pq.pop()\;
			n\_draws ++\;
			$adj = \{(nj) | \exists (nj)\in E \text{ and } \exists j \in sg_{vtx}\}$\;
			\ForAll{$(nj)\in adj$}{
				$sg = sg \cup (nj)$\;
				n\_draws = 0\;
			}
		\uIf{$|adj| < $ deg$(n)$}{
			n\_prio -= $(|adj|-1)$\;
			pq.push($n$, n\_prio)\;
		}
		\uIf{pq.size() $\leq$ n\_draws \& $\exists j |(nj) \in E, j\not\in sg_{vtx}$}{
			$j \in \left\{j |(nj) \in E, j\not\in sg_{vtx}\right\}$\;
			prio ++\;
			pq.push($j$, prio)\;
			$sg = sg \cup (nj)$)\;
			$sg_{vtx} = sg_{vtx} \cup j$\;
		}
		}
		$SG$ = $sg \cup SG$
	}
	\Return $SG$
	\caption{Dense subgraphs in a rag}
	\label{algo:sgs}
\end{algorithm}
\vspace{8mm}

\subsection{Dependence of action distributions}

The actor predicts an univariate probability distribution on each edge in the superpixel graph. Looking at the predicted statistics for this distributions as random variables, the predicted distributions themselves can be considered random variables. Since all predictions are dependent on each other through the graph convolutions there is an underlying multivariate probability distribution for the statistics on all the edges. This intuitively makes sense because actions should depend on broader local neighborhoods in the graph if not on the whole graph.\\
The same holds for the update of the critics networks. Since the subgraphs are moved into the batch dimension (see section \ref{ssec:batchp}) they have no dependence to each other through the convolution on the subgraphs. This is the reason for the upstream network doing the convolution on the whole graph. Also
the overlaps of the subgraphs help building interdependence between subgraphs.
