\chapter{Introduction}
\section{Motivation}~\label{chap:motivation}
Unsupervised learning for image learning is on a completely different level of causal inference than supervised learning. In supervised learning the labels that are provided form an intermediate step between the knowledge of the causal connection between raw data and labels, and the application of that knowledge to a learning process. Getting rid of the label generation and directly apply the necessary knowledge to the learning process would safe the whole expensive labeling process.\\
Unsupervised image segmentation algorithms like graphical models or clustering algorithms usually lack in incorporating semantic information in the image. Also classical training of CNNs is not fit for unsupervised learning because the computation that leads to the loss has to be differentiable. If for example there is enough prior knowledge on the objects like size, texture, shape ... there are yet not methods to formulate this constraints in a loss term that is differentiable.\\
Of course formulating all the prior knowledge and implementing that into a computer program can be, depending on the problem very exhaustive and is unlikely to reach completeness. However assuming it can be done, it is still not clear how to train a neural network on such rules.\\
Reinforcement learning methods have the property that it learns on temporal differences in rewards. Here the rewards can depend on the parameters of the prediction model but do not have to. This of course means that their generation does not have to be differentiable. Usually the rewards are obtained by measurements in a complex physical environment which is difficult to model.\\
This property motivates to use Reinforcement Learning in other learning problems such as the image segmentation problem.

\section{Contribution}
This work proposes a image segmentation pipeline based on Reinforcement Learning and graph neural networks where the level of supervision can be adjusted seamlessly. The pipeline has been tested on the toy problem of segmenting discs. It was tested in an almost unsupervised setting where the only supervision stems from the pretraining of a feature extractor network. The main training of the pipeline is usupervised where the non differentiable evaluation of the predictions is based on prior knowledge on the objects in the image.